
Pair programming is a software development practice
in which one person (the driver) does the typing
while the other (the navigator) offers comments and suggestions,
and the two switch roles several times per hour.
It is effective in professional work~\cite{Hann2009},
and benefits in teaching include increased success rate in introductory courses,
better software,
and higher learner confidence in their solutions.
There is also evidence that learners from underrepresented groups
benefit even more than others \cite{McDo2006,Hank2011,Cele2018}.
It is particularly helpful with mixed-ability classes,
since pairs are more homogeneous than individuals.
However,
when you use pairing,
put everyone in pairs:
if you only pair learners who are struggling,
they will feel singled out.

\emph{Peer instruction} attempts to provide the benefits of individual instruction in a scalable way
by interleaving formative assessment with small-group discussion.
After a brief introduction to a new topic,
the instructor gives learners a multiple choice question or some other formative assessment
(Tip~2).
The learners vote on their answers,
then spend several minutes discussing the question,
during which they will fill in gaps in each other's knowledge
and clear up each other's misunderstandings.
This technique makes group discussion the focus of learning,
and its effectiveness has been proven by multiple studies~\cite{Crou2001,Smit2009,Port2016}.

----


A slide deck is like taking a journey by train:
the ride may be smooth,
but the route can't be changed on the fly.
Writing code in front of the class while learners following along,
on the other hand,
is like going off road in a four-wheel drive:
it may be bumpier and messier,
but it's a lot easier to explore things that catch the learners' attention.

This practice is called \emph{live coding}.
It encourages active learning---people are using knowledge immediately
when acquiring it---and allows the instructor to adapt to their actual audience.
If a learner asks a ``what if'' question,
it's a lot easier to respond if you are writing code in real time
rather than presenting slides.
Following learners' lead also signals that you respect your learners' time and interests;
this improves engagement,
which in turn improves learning outcomes~\cite{Wlod2017}.
It also slows you down
and reminds you just how much extraneous material your learners have to wade through.

But it's possible to have too much of a good thing.
If someone asks a question, try it out.
If they or someone else asks a follow-up,
try that too,
but then come back to your main point.
Otherwise,
you can easily be pulled so far off course that
you don't reach the most important points of your lesson.

Finally,
live coding presents natural opportunities to demonstrate something that is often left out of textbooks.
Novices spend most of their time trying to figure out what's broken and how to fix it,
but lessons rarely devote that amount of time to analyzing and correcting errors.
Watching the instructor make a mistake,
figure out what went wrong,
and then make and test fixes is immediately valuable.
Doing this also shows learners that it's all right to make mistakes of their own:
if you aren't embarrassed about making and talking about mistakes,
learners will be more comfortable doing so too.

------


