\documentclass[10pt,letterpaper]{article}
\include{settings}

% Define per-paper macros.
\newcommand{\withurl}[2]{{#1}\footnote{\texttt{#2}}}
\newcommand{\rulemajor}[1]{\section{#1}}
\begin{document}
\vspace*{0.2in}

\begin{flushleft}
{\Large
\textbf\newline{Ten Quick Tips for Delivering a Programming Lesson}
}
\newline
\\
{Greg~Wilson}\textsuperscript{1,*}
\\
\textbf{1} RStudio, Inc., Toronto, Ontario M4L 2T9
\\
\bigskip
* greg.wilson@rstudio.com
\end{flushleft}

\section*{Abstract}

Designing a great lesson is the first 90\% of effective teaching;
delivering it well is the other 90\%.
The 10 simple rules outlined in this paper
describe classroom practices that instructors can adopt immediately and at low cost.

\section*{Author Summary}

Teaching well is a craft like any other,
and success often comes from an accumulation of small improvements
rather than from any single large change.
This paper describes ten practices that can be applied to a wide variety of subjects,
that are easy and inexpensive to adopt,
and that have proven their value in both regular classrooms and free-range workshops.

\section*{Introduction}

Teaching is a craft,
and like any craft it can be improved by studying, adopting, and adapting
the practices of others.
This paper describes ten practices that can be applied to a wide variety of subjects
and have proven their value in both regular classrooms and free-range workshops.
They are easier to adopt than those described in \cite{Brow2018,Deve2018},
some of which are briefly discussed in the conclusion;
some have been inspired by \cite{Hust2012,Lang2016,Lemo2014},
while others are drawn from the author's experience
teaching programming to researchers \cite{Wils2016}.

\rulemajor{Use formative assessment every 5--15 minutes.}

As instructors,
we always want to get through material than time allows,
so we often teach at the speed at which we can talk
rather than the speed at which people can learn.
Having learners do something every 5--15 slows us down,
keeps them engaged,
and gives us and them feedback on whether they have actually understood
what has just been taught.

In-class checks like this are one kind are called \emph{formative assessments}.
Good ones take only a minute or two to complete so that they don't derail the flow of the lesson,
and have an unambiguous correct answer so that they can be checked in large classes.
Popular kinds of formative assessment in programming classes include:

\begin{itemize}
  \item Answer a multiple choice question.
  \item Write a few lines of code.
  \item Predict what the code on the screen will do when it runs.
  \item Contribute the next line of code.
\end{itemize}

Starting with a formative assessment that reviews a previous lesson
is a good way to signal that class has started,
and having students recall older material before tackling something new
has been shown to improve learning outcomes \cite{Wein2018b}.
Similarly,
ending the class with such an exercise
gives learners a sense of how far they (should have) progressed.

A complement to this rule is to get students out of their seats every 45--60 minutes.
People's brains get tired when they are concentrating,
and tired brains can't learn \cite{Ambr2010,HPL2}.
Caffeine doesn't fix this,
so have your learners get up and move around for a few minutes every hour
in order to reoxygenate their gray cells.
This allows those who need a bathroom break to take care of things discreetly\footnote{A colleague once told me that
the basic unit of teaching is the bladder.
When I said I'd never thought of that,
she said, ``You've obviously never been pregnant.''},
and give the instructor a few moments to answer questions
and figure out what to teach next.

\rulemajor{Monitor progress and learners' need for assistance.}

Sticky notes are one of my favorite teaching tools,
and I'm not alone in loving their versatility,
portability, stickability, foldability,
and subtle yet alluring aroma \cite{Ward2015}.
Give each student two sticky notes of different colors,
such as orange and green.
If someone has completed an exercise and wants it checked,
or if they feel that they are following the lesson,
they put the green sticky note somewhere the teacher can see.
If they run into a problem and need help,
they put up the orange one.
This works much better than having people raise their hands:
it's more discreet (which means they're more likely to actually do it),
they can keep working while their flag is raised rather than trying to type one-handed,
and the teacher can quickly see from the front of the room what state the class is in.

\rulemajor{Use a visibly fair mechanism to distribute your attention evenly.}

Teachers naturally focus their attention on learners
who are making eye contact and asking lots of questions---in other words, on extroverts.
This creates two feedback loops:
the extroverts ask even more questions because they're getting attention,
while other students stop trying to engage because they're not.
To ensure the teacher's attention is fairly distributed,
have each learner write their name on a sticky note
and post it somewhere visible.
Each time the teacher calls on them or answers one of their questions,
they take their sticky note down.
Once all the sticky notes are down,
everyone puts theirs up again.
This technique makes it easy for the teacher to see who they haven't spoken with recently,
which helps them avoid unconscious bias and preferential interaction.
It also shows learners how attention is being distributed
so that when they \emph{are} called on,
they won't feel like they're being picked on.

\rulemajor{Make mistakes and work through them.}

Novices spend most of their time trying to figure out what's broken and how to fix it,
but teachers rarely devote that proportion of lessons to analyzing and correcting errors.
Watching you figure out that something is wrong,
hearing you work backward through the error messages to possible causes,
and seeing you make and test fixes
is often one of the most valuable things you can show your learners.

For example,
an effective way to teach programming is \emph{live coding}.
Instead of presenting pre-written material,
teachers write code in front of their class as their learners follow along.
Doing this provides lots of opportunities for making and correcting mistakes,
and watching a program being written is more engaging than watching someone page through slides.
It also slows the teacher down,
and helps remind them just how much they are throwing at their learners.
Finally,
watching teachers make mistakes shows learners that it's all right to make mistakes of their own.
If the teacher isn't embarrassed about making and talking about mistakes,
learners will be more comfortable doing so too.

\rulemajor{Follow your learners---twice.}

A slide deck is like taking a journey by train:
the ride may be smooth,
but the route can't be changed on the fly.
Live coding,
on the other hand,
is like going off road in a four-wheel drive:
it may be bumpier and messier,
but it's a lot easier to explore things that catch your attention.
In particular,
if a student asks a ``what if'' question,
it's a lot easier to respond if you are writing code or proving a theorem in real time
than if you are using a slide deck.
Following your learners' lead also signals that you respect their time and interests;
this improves engagement,
which in turn improves learning outcomes \cite{Wlod2017}.

But it's possible to have too much of a good thing.
If someone asks a question, try it out.
If they or someone else asks a follow-up,
try that too,
but then come back to your main point.
Otherwise,
you can easily find yourself pulled so far off course that
you don't reach the most important points of your lesson.

\rulemajor{Create a backlog.}

You may not have time to answer all of your students' questions,
or might not actually know the answers.
To handle this,
write questions on sticky notes and post them on the wall behind you,
then look over this backlog during breaks and decide which questions you want to tackle.
This gives you a chance to prioritize based on what's most relevant (and what you actually know).
It also helps build trust:
many people have learned that ``I'll address that later''
means ``I hope you forget that you asked'',
so if they see you trying to tackle a few of the questions that have come up,
they will forgive you for not getting to the rest.

\rulemajor{Have learners work in pairs.}

A sense of community is one of the primary motivators for adult learning \cite{Wlod2017},
so encourage your learners to talk to teach as they work through exercises.
Peer instruction puts group discussion at the center of teaching \cite{Crou2001,Smit2009,Port2016},
but \emph{pair programming} and similar practices are also effective.

Pair programming is a software development practice
in which one person (the driver) does the typing
while the other (the navigator) offers comments and suggestions,
and the two switch roles several times per hour.
It is effective in professional work \cite{Hann2009},
and benefits in teaching include increased success rate in introductory courses,
better software,
and higher student confidence in their solutions.
There is also evidence that students from underrepresented groups
benefit even more than others \cite{McDo2006,Hank2011,Cele2018}.
I have found it particularly helpful with mixed-ability classes,
since pairs are more homogeneous than individuals.
When you use pairing,
put everyone in pairs,
not just learners who are struggling,
so that no one feels singled out.

\rulemajor{Present diagrams incrementally.}

Our brains have separate channels for
processing visual and linguistic information \cite{Ambr2010,HPL2},
so people learn best when complementary material is presented simultaneously through these channels.
In simple terms,
that means your slides should present diagrams or other relevant images for you to talk about,
and that these diagrams should be built up piece by piece rather than presented all at once.
When the diagram is presented piece by piece,
students' brains will correlate the arrival of new visual information
with the arrival of new linguistic information \cite{Maye2003,Maye2009}.
Presentation of either later on will then help trigger recall of the other.

\rulemajor{Have learners take notes, and review them.}

Taking notes forces students to organize and reflect on information as it's coming in,
which in turn increases the likelihood that you will transfer it to long-term memory.
Many studies have shown that note-taking improves retention \cite{Aike1975,Boha2011},
and more recent results indicate that taking notes using pen and paper
is more effective than using a tablet or computer \cite{Muel2014}.

Fifty years ago,
when being able to accurately record information or the minutes of a meeting
was considered an essential white-collar skill,
it was common for high school teachers to require students to submit their notes for grading.
This is less common today,
but is still beneficial,
since it gives the teacher an opportunity to identify gaps between
what they thought they taught
and what their students heard.

If the resources to do this are not available,
have learners take a minute at the end of each class
to write one thing they learned on one side of a card
and one question they still have or something they're confused about on the other.
Review these \emph{minute cards} and looking for patterns only takes a few minutes,
and tells you what you need to clarify at the start of the next lesson.

\rulemajor{Teach together when you can.}

\emph{Co-teaching} describes any situation
in which two teachers work together in the same classroom.
\cite{Frie2016} describes several ways to do this:

\begin{description}

\item[Team teaching:]
  Both teachers deliver a single stream of content in tandem,
  taking turns like musicians taking solos.

\item[Teach and assist:]
  Teacher A teaches while Teacher B moves around the classroom
  to help struggling students.

\item[Alternative teaching:]
  Teacher A provides a small set of students with more intensive or specialized instruction
  while Teacher B delivers a general lesson to the main group.

\item[Teach and observe:]
  Teacher A teaches while Teacher B observes the students,
  collecting data on their understanding to help plan future lessons.

\item[Parallel teaching:]
  The class is divided in two
  and the teachers present the same material simultaneously to each.

\item[Station teaching:]
  The students are divided into small groups
  that rotate from one station or activity to the next
  while teachers supervise where needed.

\end{description}

Team teaching is particularly beneficial in day-long workshops:
it give each teacher a chance to rest and think about what they are going to do next.
If you and a partner are co-teaching:

\begin{itemize}

\item
  Take 2--3 minutes before the start of each class
  to confirm who's teaching what.

\item
  Work out a couple of hand signals as well.
  ``You're going too fast,''
  ``speak up,''
  ``that learner needs help,''
  and, ``It's time for a bathroom break'' are all useful.

\item
  Each person should teach for at least 10--15 minutes at a stretch,
  since students will be distracted by more frequent switch-ups.

\item
  The person who isn't teaching shouldn't interrupt,
  offer corrections or elaborations,
  or do anything else to distract from what the person teaching is doing or saying,
  but may ask leading questions
  if the learners seem lethargic or unsure of themselves.

\end{itemize}

Most importantly,
take a few minutes when the class is over to congratulate or commiserate with each other:
in teaching as in life,
shared misery is lessened and shared joy increased.

\section*{Conclusion}

If you are teaching an evening class after working for a full day,
you and your learners will both appreciate it if you brush your teeth and put on a clean shirt.
Cough drops will help you keep your voice
and fend off whatever colds the learners brought with them,
and your back will be grateful tomorrow that you wore comfortable shoes today.

\bibliography{10-deliver}

\end{document}
